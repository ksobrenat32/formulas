\documentclass[letterpaper]{article}

% --- Packages
\usepackage[utf8]{inputenc}
\usepackage[T1]{fontenc}
\usepackage[margin=0.25cm]{geometry}
\usepackage{enumitem}
\usepackage{pdfpages}
\usepackage{multicol}
\usepackage{amsmath}
\usepackage{amssymb}
\usepackage[skip=1pt plus1pt, indent=0pt]{parskip}
\usepackage{enumitem}
\usepackage{graphicx}

% --- Data
\title{Fourier analysis}
\author{Enrique Calderon}
\date{November 2023}

% --- Graphics path
\graphicspath{ {./img/} }

% --- Custom commands
\makeatletter
\let\thetitle\@title
\let\theauthor\@author
\makeatother
\newcommand{\compconj}[1]{%
	\overline{#1}%
}
\newcommand{\divline}{\noindent\makebox[\linewidth]{\rule{\textwidth}{0.4pt}}}
\newcommand{\taninv}{\tan^{-1}}

\begin{document}
        \maketitle

        \divline

	\begin{multicols}{2}
		\section{Fourier}

                \textbf{Main formula}
  
                \[f(t) = a_{0} + \sum_{n = 1}^{\infty} a_{n} \cos{( \frac{2 \pi n}{T} t )} + \sum_{n = 1}^{\infty} b_{n} \sin{( \frac{2 \pi n}{T} t )} \]

                \textbf{Orthogonality}

                \[ \int_{-\frac{T}{2}}^{\frac{T}{2}} f_{1}(t) f_{2}(t) dt = 
                    \begin{cases} 
					\alpha \text{ if } f_{1} = f_{2} \\
					0 \text{ if } f_{1} \neq f_{2}
				\end{cases}
                \]

                \textbf{Orthonormality}

                \[ \int_{-\frac{T}{2}}^{\frac{T}{2}} (f_{1}(t))^2 dt = 1 \]

                The oscillation is named Gibbs phenomenon.
                The equality does not always maintain when operating in both sides.
                
	\end{multicols}
 
	\divline

        \begin{multicols}{2}
            \section{Sin and cosine}

                Using \(\frac{2}{T}\) to normalize:

                \textbf{cos \(\cdot\) cos}

                    \[ \frac{2}{T} \int_{-\frac{T}{2}}^{\frac{T}{2}} \cos{( \frac{2 n \pi}{T} t )} \cos{( \frac{2 m \pi}{T} t )} dt =
                        \begin{cases} 
					   0 \text{ if } n \neq 0 \\
					   1 \text{ if } n = 0
				    \end{cases}
                    \]

                \textbf{sin \(\cdot\) cos}

                    \[ \frac{2}{T} \int_{-\frac{T}{2}}^{\frac{T}{2}} \sin{( \frac{2 n \pi}{T} t )} \cos{( \frac{2 m \pi}{T} t )} dt = 0\]

                \textbf{sin \(\cdot\) sin}

                    \[ \frac{2}{T} \int_{-\frac{T}{2}}^{\frac{T}{2}} \sin{( \frac{2 n \pi}{T} t )} \sin{( \frac{2 m \pi}{T} t )} dt =
                        \begin{cases} 
					   0 \text{ if } n \neq 0 \\
					   1 \text{ if } n = 0
				    \end{cases}
                    \]

                \textbf{Applied in Fourier}

                    \[ \int_{-\frac{T}{2}}^{\frac{T}{2}} f(t) \cos{( \frac{2 n \pi}{T} t )} dt = \sum_{n = 1}^{\infty} a_{n} \cdot \frac{T}{2} \]

                    \[ \int_{-\frac{T}{2}}^{\frac{T}{2}} f(t) \sin{( \frac{2 n \pi}{T} t )} dt = \sum_{n = 1}^{\infty} b_{n} \cdot \frac{T}{2} \]

        \end{multicols}

        \divline

        \begin{multicols}{2}

            \section{Geometric Fourier series}

                \[f(t) = a_{0} + \sum_{n = 1}^{\infty} a_{n} \cos{( \frac{2 \pi n}{T} t )} + \sum_{n = 1}^{\infty} b_{n} \sin{( \frac{2 \pi n}{T} t )} \]

                Where:

                \[a_{0} = \frac{1}{T} \int_{-\frac{T}{2}}^{\frac{T}{2}} f(t) dt \]

                \(a_{0}\) is the average value.

                \[a_{n} = \frac{2}{T} \int_{-\frac{T}{2}}^{\frac{T}{2}} f(t) \cos{( \frac{2 \pi n}{T} t )} dt \]

                \[b_{n} = \frac{2}{T} \int_{-\frac{T}{2}}^{\frac{T}{2}} f(t) \sin{( \frac{2 \pi n}{T} t )} dt \]

        \end{multicols}

        \divline

        \begin{multicols}{3}
            \section{Parity functions}
                
                \textbf{Even:}

                \[f(t) = f(-t)\]

                For integrals in origin you can calculate half and multiply by 2.

                \textbf{Odd:}

                \[f(t) = - f(-t)\]

                Integrals in origin are equal to 0.

                \textbf{Properties:}

                \[f_{p}(t) \cdot f_{p}(t) = f_{p}(t)\]
                \[f_{i}(t) \cdot f_{i}(t) = f_{p}(t)\]
                \[f_{i}(t) \cdot f_{p}(t) = f_{i}(t)\]

                \textbf{In Fourier}

                If \(f(t)\) is even:

                \[a_{0} = \frac{2}{T} \int_{0}^{\frac{T}{2}} f_{p}(t) dt \]
                \[a_{n} = \frac{4}{T} \int_{0}^{\frac{T}{2}} f_{p}(t) \cos{(\frac{2 n \pi}{T} t)} dt \]
                \[b_{n} = 0\]

                If \(f(t)\) is odd:

                \[a_{0} = 0\]
                \[a_{n} = 0\]
                \[b_{n} = \frac{4}{T} \int_{0}^{\frac{T}{2}} f_{p}(t) \sin{(\frac{2 n \pi}{T} t)} dt \]
                
        \end{multicols}

        \divline

        \begin{multicols}{2}

            \section{Complex Fourier series}

                \[f(t) = \sum_{n = - \infty}^{\infty} C_{n} e^{\frac{i2n\pi}{T} t} \]

                Where:

                \[C_{n} = \frac{1}{T} \int_{-\frac{T}{2}}^{\frac{T}{2}} f(t) e^{-\frac{i2n\pi}{T}t} dt \]

                \[a_{n} = 2 \mathbb{R}\text{e}(C_{n})\]
                \[b_{n} = - 2 \mathbb{I}\text{m}(C_{n})\]

                Amplitude spectrum: The value of \(|C_{n}|\) as \(y\) and \((2\pi/T)n\) as \(x\).

                Phase spectrum: The value of \(arg(Cn)\) as \(y\) and \((2\pi/T)n\) as \(x\).

        \end{multicols}

        \divline

        \begin{multicols}{2}

            \section{Fourier transform}

                \[\mathcal{F} (f(t)) = F(W) = \frac{1}{2\pi} \int_{-\infty}^{\infty} f(t) e^{-iwt} dt \]
                \[\mathcal{F}^{-1} (F(W)) = f(t) = \int_{-\infty}^{\infty} F(W) e^{iwt} dw \]
        
        \end{multicols}

        \divline

        \begin{multicols}{2}
            \section{Trigonometric integral}

                \[F(W) = \int_{-\infty}^{\infty} f(t) \cos(wt) dt -i \int_{-\infty}^{\infty} f(t) \sin(wt) dt \]

                \[F_{c}(W) = \int_{-\infty}^{\infty} f(t) \cos(wt) dt  \]
                
                \[F_{s}(W) = \int_{-\infty}^{\infty} f(t) \sin(wt) dt  \]

                \[f(t) = \frac{1}{2\pi} \int_{-\infty}^{\infty} [F_{c}(W) - i F_{s}(W)][\cos(wt) + i \sin(wt)] dt \]

        \end{multicols}

        \divline

        \begin{multicols}{2}
            \section{Steps to solve Fourier transform with tables}

                \begin{enumerate}
                    \item Identify the properties that can be applied.
                    \item Algebraic work to make the problem look as a propertie.
                    \item The properties should reduce the complexity of the problem to solve.
                \end{enumerate}

        \end{multicols}

        \divline

        \begin{multicols}{2}
            \section{Convolution}

                \[f_{1}(t) * f_{2}(t) = \int_{-\infty}^{\infty} f_{1}(x)f_{2}(t-x)dx \]
                
                \[f_{1}(t) * H(t) = \int_{-\infty}^{\infty} f(t-x)dx \]

                \[f_{1}(t) * \delta(t) = f(t) \]

                \[f_{1}(t) * f_{2}(t) = \mathcal{F}_{1}(W) \mathcal{F}_{2}(W) \]

                \[f_{1}(t) f_{2}(t) = \frac{1}{2\pi} [\mathcal{F}_{1}(W) * \mathcal{F}_{2}(W)] \]
        
        \end{multicols}
        
        \divline

        \begin{multicols}{2}
            \section{Miscellaneous}

                \[\text{sinc} = \frac{\sin(x)}{x}\]

                \[H(t) = u(t)\]

                If you reduce time the function goes to the right, if you add time the function goes to the left.

                \[t^{n} \cdot f(t) = \frac{1}{(-i)^{n}} \cdot \frac{\delta^{n}}{dw^{n}} F(W)\]
        
        \end{multicols}
 
	\divline
 
\end{document}
